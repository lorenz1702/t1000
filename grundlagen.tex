


\chapter{Grundlagen}

\section{Appstudio}
SICK AppSpace - Engineering-Framework für Ihre individuellen Sensoranwendungen

SICK AppSpace ist ein Ökosystem rund um programmierbare Sensoren und Geräte von SICK und individualisierte SensorApps. Als Teil einer dynamischen Entwickler-Community können Kunden eigenständig oder gemeinsam mit den Experten von SICK SensorApps erstellen. Mit diesen SensorApps lassen sich alle Anwendungen mit unterschiedlichsten Technologien lösen. Individualisierte SensorApps werden mit intelligenten Softwaretools und Algorithmen erstellt. Bestehende Lösungen für Track & Trace, Positionieraufgaben, Roboterführung oder Qualitätskontrolle können an die individuellen Bedürfnisse der Kunden angepasst werden; und völlig neue SensorApps können nach individuellen Anforderungen und absolut maßgeschneidert für bestehende Systeme erstellt werden. SICK AppSpace unterstützt heute eine Reihe von Geräten und Technologien, wie 2D-Vision, 3D-Vision, LiDAR, RFID oder Integrationsprodukte. Das SICK-AppSpace-Ökosystem besteht aus drei Hauptkomponenten. Zum einen aus der Hardware, das heißt programmierbare Sensoren, Sensorintegrationsmaschinen und andere Geräte. Zum anderen Software und Tools, das heißt die Werkzeuge zum Erstellen, Verteilen und Bereitstellen von SensorApps. Und als letzten Hauptkomponenten die Community, das heißt die Mitglieder des SICK AppSpace-Entwicklerclubs, die sich im SICK Support Portal und Konferenzen austauschen.

Die programmierbaren Geräte im SICK-AppSpace-Ökosystem bieten Raum für die Integration der kundenspezifischen SensorApps und ermöglichen so maßgeschneiderte Applikationslösungen. So werden je nach Anwendung völlig neue Lösungen im Bereich der Automatisierung möglich - und die SensorApps können jederzeit angepasst oder ausgetauscht werden. SICK AppSpace-Sensoren und -Geräte bieten eine "Dual Talk"-Funktion, die eine gleichzeitige Kommunikation mit einer SPS sowie mit übergeordneten Enterprise-Level-Systemen und sogar Cloud-Services ermöglicht. Sie unterstützen damit den Aspekt der vertikalen Integration von Industrie 4.0.

Das AppEngine-Framework ist das Herzstück der Firmware bei allen programmierbaren Geräten. Es bietet eine umfangreiche Anwendungsprogrammierschnittstelle (SICK Algorithm API) mit einem breiten, gerätespezifischen Satz an vordefinierten Funktionen und Operatoren. Geräte mit Bildverarbeitung bieten zwei Methoden der Bildverarbeitung - entweder mit den 2D- und 3D-Operatoren der SICK Algorithm API oder mit der integrierten leistungsfähigen HALCON-Vision-Bibliothek.

Die zentralen Werkzeuge im SICK AppSpace-Ökosystem sind das SICK AppStudio, der SICK AppManager und der SICK AppPool.


Das SICK AppStudio ist eine integrierte Anwendungsentwicklungsumgebung zur Erstellung von SensorApps. Mit diesen SensorApps können Entwickler kundenspezifische Applikationslösungen auf programmierbaren Geräten erstellen. SensorApps können mit Standardkomponenten und Programmiersprachen programmiert werden. Anwendungsentwickler mit weniger Interesse am Programmieren können auch Datenflüsse modellieren, wobei ein tieferes Eintauchen in die Programmierung immer möglich ist. Die Benutzeroberflächen der SensorApps sind webbasiert, so dass sie in jedem Browser angezeigt werden können. Sie können vom SensorApp-Entwickler über einen grafischen UI-Builder oder mit Standard-Webtechnologien individuell gestaltet werden.

 

SensorApps werden mit dem Deployment-Tool SICK AppManager installiert, aktualisiert und verwaltet. Es integriert sich direkt in den SICK AppPool und unterstützt das automatische Deployment über ein CLI. Der vollständige SICK AppManager ist kostenlos und für Windows (x64) verfügbar, die reine CLI-Version ist auch für Windows-32bit-, Linux-64bit- und ARM-32bit-Systeme verfügbar.

 

Der SICK AppPool ist der zentrale und sichere Cloud-Service zum Austausch von SensorApps. SensorApp-Entwickler können ihre SensorApps veröffentlichen und entweder mit einer ausgewählten Gruppe von Nutzern oder mit der ganzen Welt teilen. Interessierte können SensorApps nach Stichworten, kompatiblen Geräten, Applikationen, Herausgebern und zahlreichen weiteren Filtern finden. Die Veröffentlichung im SICK AppPool ist ein Privileg der Mitglieder des SICK AppSpace Developers Club.

 

Um SensorApps zu entwickeln, ist eine Mitgliedschaft im SICK AppSpace Developers Club erforderlich. Diese Mitgliedschaft beinhaltet eine Volllizenz für das SICK AppStudio. Für die Dauer von einem Jahr haben Mitglieder außerdem Zugriff auf das Ticket-Supportsystem im SICK Support Portal, Entwicklerschulungen, exklusive Downloads und viele weitere Vorteile. Darüber hinaus werden sie zu den jährlichen SICK AppSpace-Entwicklerkonferenzen und anderen Veranstaltungen eingeladen, wo sie ihre Arbeit präsentieren, Ideen austauschen und die Entwicklung des SICK AppSpace-Ökosystems mitgestalten können.

\section{LUA}

Lua ist eine imperative und erweiterbare Skriptsprache zum Einbinden in Programme, um diese leichter weiterentwickeln und warten zu können. Eine der besonderen Eigenschaften von Lua ist die geringe Größe des kompilierten Skript-Interpreters.

Lua-Programme sind meist plattformunabhängig und werden vor der Ausführung in Bytecode übersetzt. Obwohl man mit Lua auch eigenständige Programme schreiben kann, ist sie vorrangig als eingebettete Skriptsprache für andere Programme konzipiert. Vorteile von Lua sind die geringe Größe von 120 kB, die Erweiterbarkeit und die hohe Geschwindigkeit, verglichen mit anderen Skriptsprachen.

Der Lua-Interpreter kann über eine C-Bibliothek angesprochen werden, die auch ein API\footnote{von englisch application programming interface, wörtlich ‚Anwendungs­programmier­schnittstelle} für die Laufzeitumgebung des Interpreters für Aufrufe vom C-Programm aus enthält. Mittels des API können verschiedene Teile des Programmes in C (oder C++) und Lua geschrieben werden, während Variablen und Funktionen in beiden Richtungen erreichbar bleiben (d. h. eine Funktion in Lua kann eine Funktion in C/C++ aufrufen und umgekehrt).

\section{LMS4000}

\section{Encoder}

\section{Clean Coding}


